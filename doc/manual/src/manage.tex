\chapter{Cluster Management}

\section{Local node status and configuration}
You can check status of a specific node by running \path{sxserver status} on
that node:
\begin{lstlisting}
# sxserver status
--- SX STATUS ---
sx.fcgi is running (PID 14394)
sxhttpd is running (PID 14407)
\end{lstlisting}
Run \path{sxsetup --info} to display the node's configuration:
\begin{lstlisting}
# sxsetup --info
--- SX INFO ---
SX Version: 2.0
Cluster name: ^\marked{mycluster}^
Cluster port: 443
Cluster UUID: 01dca714-8cc9-4e26-960e-daf04892b1e2
Cluster key: CLUSTER/ALLNODE/ROOT/USERwBdjfz3tKcnTF2ouWIkTipreYuYjAAA
Admin key: ^\marked{0DPiKuNIrrVmD8IUCuw1hQxNqZfIkCY+oKwxi5zHSPn5y0SOi3IMawAA}^
Internal cluster protocol: SECURE
Used disk space: 16.75M
Actual data size: 453.00K
List of nodes:
         * ec4d9d63-9fa3-4d45-838d-3e521f124ed3 ^\marked{192.168.1.101}^ (192.168.1.101) 500.00G
Storage location: /var/lib/sxserver/storage
SSL private key: /etc/ssl/private/sxkey.pem
SX Logfile: /var/log/sxserver/sxfcgi.log
\end{lstlisting}
This gives you the information about local services and disk usage, but
also provides the admin key, which is needed for accessing the cluster
itself.

\section{Administrator access}
During cluster deployment a default admin account gets created
and initialized. For security reasons, the account uses a randomly
generated key instead of a password. You should be able to access the
cluster from any node using \path{sx://admin@mycluster} profile. In order
to manage the cluster remotely or from another system account,
you need to initialize access to the cluster using \path{sxinit}
\footnote{For more information about access profiles please see
{\ifpdf\fref{sec:profiles}\else\ref{sec:profiles}\fi}.}.
In the example below we use the default admin account created
during cluster setup. Since ``mycluster'' is not a DNS name, we need
to point \path{sxinit} to one of the nodes of the cluster --- this will
allow it automatically discover the IP addresses of the other nodes.
Additionally, we create an alias \path{@cluster}, which later
can be used instead of \path{sx://admin@mycluster}.
\begin{lstlisting}
$ sxinit --key -l 192.168.1.101 -A @cluster sx://admin@mycluster
Warning: self-signed certificate:

        Subject: C=GB, ST=UK, O=SX, CN=mycluster
	Issuer: C=GB, ST=UK, O=SX, CN=mycluster
	SHA1 Fingerprint: 84:EF:39:80:1E:28:9C:4A:C8:80:E6:56:57:A4:CD:64:2E:23:99:7A

Do you trust this SSL certificate? [y/N] ^\marked{y}^
Trusting self-signed certificate
Please enter the user key: ^\marked{0DPiKuNIrrVmD8IUCuw1hQxNqZfIkCY+oKwxi5zHSPn5y0SOi3IMawAA}^
\end{lstlisting}

\section{User management}
\SX similarly to UNIX systems supports two types of users: regular and
administrators. A new cluster has only a single `admin' user enabled by
default. The administrators can perform all cluster operations and access
all data in the cluster (except for encrypted volumes), while the regular
users can only work with volumes they have access to. It is recommended to
only use the admin account for administrative purposes and perform regular
operations as a normal user.

\subsection{Creating a new user}
Use \path{sxacl useradd} to add new users to the cluster:
\begin{lstlisting}
$ sxacl useradd jeff @cluster
Enter password for user 'jeff'
Enter password:
Re-enter password: 
User successfully created!
Name: jeff
Key : FqmlTd9CWZUuPBGMdjE46DaT1/3kx+EYbahlrhcdVpy/9ePfrtWCIgAA
Type: normal

Run 'sxinit sx://jeff@mycluster' to start using the cluster as user 'jeff'.
\end{lstlisting}
By default a regular user account gets created and the key is generated from the
password. The user can later authenticate both using the password or the key.
It's also possible to automatically generate a random key by passing the
\verb+--generate-key+ option.

\subsection{Listing users}
In order to list all users in the cluster run:
\begin{lstlisting}
$ sxacl userlist @cluster
admin (admin)
jeff (normal)
\end{lstlisting}
Only cluster administrators can list users.

\subsection{Key and password management}
\SX uses special authentication keys, which are either randomly
generated or based on passwords. It is possible to obtain the
existing key or issue a new one for any user in the cluster.
To retrieve the current authentication key for user `jeff' run:
\begin{lstlisting}
$ sxacl usergetkey jeff @cluster
5tJdVr+RSpA/IPuFeSwUeePtKdbDLWUKqoaoZLkmCcXTw5qzPg5e7AAA
\end{lstlisting}
A new password/key can be set at any time by running:
\begin{lstlisting}
$ sxacl usernewkey jeff @sxtest
Enter new password for user 'jeff'
Enter password:
Re-enter password: 
Key successfully changed!
Name   : jeff
New key: FqmlTd9CWZUuPBGMdjE46DaT1/3MSHk9TLH27dFf5Zd61lEbWEeAqgAA
Run 'sxinit sx://jeff@sxtest' and provide the new key for user 'jeff'.
\end{lstlisting}
As long as the user can access the cluster, it can change its own
key. The cluster administrator can force a key change for any user,
what can also be used to temporarily block access to the cluster for
a specified user.

\subsection{Removing a user}
Use \path{sxacl userdel} to permanently delete a user from the cluster:
\begin{lstlisting}
$ sxacl userdel jeff @cluster
User 'jeff' successfully removed.
\end{lstlisting}
All volumes owned by the user will be automatically reassigned to the
cluster administrator performing the removal.

\section{Volume management} \label{sec:volumes}
Volumes are logical partitions of the \SX storage, which are of
specific size and accessible by a particular group of users. The
volumes can be used in connection with client side filters to perform
additional operations, such as compression or encryption.
Only cluster administrators can create and remove volumes.

\subsection{Creating a plain volume}
Below we create a basic volume of size 50GB owned by the user `jeff' and fully replicated on two nodes.
\begin{lstlisting}
$ sxvol create -o jeff -r 2 -s 50G @cluster/vol-jeff
Volume 'vol-jeff' (replica: 2, size: 50G, max-revisions: 1) created.
\end{lstlisting}
By default, a volume will only keep a single revision of each file (\path{max-revisions}
parameter set to 1). The revisions are previous versions of the file stored when the file
gets modified. For example, when a volume gets created with \path{max-revisions} set to
3, and some file gets modified multiple times, then the latest 3 versions of the file will
be preserved. All revisions are accounted for their size. See the information about
\path{sxrev} in {\ifpdf\fref{sec:files}\else\ref{sec:files}\fi} on how to manage file
revisions.

\subsection{Creating a filtered volume}
Filters are client side plugins, which perform operations on files or their contents, before
and after they get transferred to/from the \SX cluster. When a filter gets assigned to a volume,
all remote clients will be required to have support for that particular filter in order to access the volume.
Run the following command to list the available filters:
\begin{lstlisting}
$ sxvol filter --list
Name            Ver     Type        Short description
----            ---     ----        -----------------
undelete        1.2     generic     Backup removed files
zcomp           1.2     compress    Zlib Compression Filter
aes256          2.0     crypt	    Encrypt data using AES-256-CBC-HMAC-512
attribs         1.3     generic     File Attributes
\end{lstlisting}
We will create an encrypted volume for user `jeff'. To obtain more information
about the \path{aes256} filter run:
\begin{lstlisting}
$ sxvol filter -i aes256
'aes256' filter details:
Short description: Encrypt data using AES-256-CBC-HMAC-512 mode.
Summary: The filter automatically encrypts and decrypts all data using
	 OpenSSL's AES-256 in CBC-HMAC-512 mode.
Options: 
        setkey (set a permanent key when creating a volume)
	paranoid (don't use key files at all - always ask for a password)
	encrypt_filenames: enable encryption of filenames (may be slow with big number of files)
	salt:HEX (force given salt, HEX must be 32 chars long)
UUID: 15b0ac3c-404f-481e-bc98-6598e4577bbd`
Type: crypt
Version: 2.0
\end{lstlisting}
Now run the following command to create an encrypted volume:
\begin{lstlisting}
$ sxvol create -o jeff -r 2 -s 50G -f aes256 @cluster/vol-jeff-aes
Volume 'vol-jeff-aes' (replica: 2, size: 50G, max-revisions: 1) created.
\end{lstlisting}
The user will be asked to set a new password while accessing the volume
for the first time.

\subsection{Listing all volumes}
To get a list of all volumes in the cluster run \path{sxls} with the cluster
argument as an administrator. When the same command is run by a normal user,
it will list all volumes, to which the user has access.
\begin{lstlisting}
$ sxls -lH @cluster
  VOL  rep:2  rev:1  rw  -       0  50.00G  0% sx://admin@mycluster/vol-jeff
  VOL  rep:2  rev:1  rw  aes256  0  50.00G  0% sx://admin@mycluster/vol-jeff-aes
\end{lstlisting}
The \path{-l (--long-format)} flag makes \path{sxls} provide more information
about the volumes, and \path{-H} converts all sizes into a human readable form.
The parameters right after the volume marker \path{VOL} are: number of replicas,
maximum number of revisions, access permissions for the user performing the
listing (in this case for the administrator), active filter, used space, size
of the volume, and the usage percentage.

\subsection{Managing volume permissions}
Cluster administrators and volume owners can grant or revoke access
to the volumes to other users. The owner can also grant another user
the manager privilege, which allows to manage the volume permissions.
To list the current access control list for the volume \path{vol-jeff} run:
\begin{lstlisting}
$ sxacl volshow @cluster/vol-jeff
admin: read write
jeff: read write manager owner
(all admin users): read write admin
\end{lstlisting}
To grant full access to user `bob' run:
\begin{lstlisting}
$ sxacl volperm --grant=read,write bob @cluster/vol-jeff
New volume ACL:
admin: read write
bob: read write
jeff: read write manager owner
(all admin users): read write admin
\end{lstlisting}
User `bob' can now upload, download and remove files from the volume but cannot
make any changes to the volume settings (this is restricted to admins, managers
and owners). To revoke write access from user `bob' run:
\begin{lstlisting}
$ sxacl volperm --revoke=write bob @cluster/vol-jeff
New volume ACL:
admin: read write
bob: read
jeff: read write manager owner
(all admin users): read write admin
\end{lstlisting}
Now `bob' can only read files but cannot upload or remove anything.

\subsection{Changing volume settings}
Some of the volume settings such as its size or ownership can be
modified at a later time. For example, the cluster administrator may
want to extend a volume size or shrink it to forbid users from storing
more data --- when the new size is lower than the current space usage of
the volume the existing contents will remain untouched --- but in order
to upload more data to the volume, the user will have to make enough
space to satisfy the new limit.

To resize the volume `vol-jeff' to 100GB run:
\begin{lstlisting}
$ sxvol modify --size 100G @cluster/vol-jeff
\end{lstlisting}

\section{Node management}
In {\ifpdf\fref{sec:addnode}\else\ref{sec:addnode}\fi} we described how to
add new nodes to a cluster. This
section covers other modifications to an existing cluster, such as node
repair, resize or delete. In the examples below we will manage a cluster
with four nodes, 500GB each, with an administrator profile configured
as \path{@cluster2}.

\subsection{Remote cluster status}
To get information about remote cluster status run the following command:
\begin{lstlisting}
$ sxadm cluster --info @cluster2
Cluster UUID: cc8ab859-619e-4806-ade6-c32ab2db1665
Operating mode: read-write
Current configuration: 536870912000/192.168.100.1/d3f8ad83-d003-4aaa-bbfb-73359af85991 536870912000/192.168.100.2/abc2ed51-b4a8-46b6-a8ac-0beb58e697d2 536870912000/192.168.100.3/a343b7f9-0bef-4f03-8c6f-526ca12d75a9 536870912000/192.168.100.4/b9b05fc7-7a4b-417d-853b-ac56ed32f5d3 
Distribution: 872eeecb-ebf9-4368-8150-beb23cd44edf(v.7) - checksum: 18024964248989723179
State of nodes:
  * node d3f8ad83-d003-4aaa-bbfb-73359af85991: addr: 192.168.100.1, capacity: 536870912000, status: follower, online: yes
  * node abc2ed51-b4a8-46b6-a8ac-0beb58e697d2: addr: 192.168.100.2, capacity: 536870912000, status: follower, online: yes
  * node a343b7f9-0bef-4f03-8c6f-526ca12d75a9: addr: 192.168.100.3, capacity: 536870912000, status: leader, online: yes
  * node b9b05fc7-7a4b-417d-853b-ac56ed32f5d3: addr: 192.168.100.4, capacity: 536870912000, status: follower, online: yes
\end{lstlisting}
The first line provides the list of cluster nodes in the following format:
\begin{lstlisting}
SIZE/IP_ADDRESS[/INTERNAL_IP_ADDRESS]/UUID
\end{lstlisting}
where \path{SIZE} is in bytes and \path{UUID} is a unique identifier assigned to a node
when joining the cluster.

In order to get information about the individual nodes run (\verb+-H+ converts
sizes to human readable form):
\begin{lstlisting}
$ sxadm cluster --list-nodes -H @cluster2
Node d3f8ad83-d003-4aaa-bbfb-73359af85991 status:
    Versions:
        SX: 2.0
        HashFS: SX-Storage 2.0
    System:
        Name: Linux
        Architecture: x86_64
        Release: 3.2.0-4-amd64
        Version: #1 SMP Debian 3.2.51-1
        CPU(s): 8
        Endianness: little-endian
        Local time: 2015-05-07 17:03:21 CEST
        UTC time: 2015-05-07 15:03:21 UTC
    Network:
        Public address: 192.168.100.1
        Internal address: 192.168.100.1
    Storage:
        Storage directory: /var/lib/sxserver/storage
        Allocated space: 259.47G
        Used space: 227.30G
    Storage filesystem:
        Block size: 4.00K
        Total size: 1.14T
        Available: 644.28G
        Used: 53.28%
    Memory:
        Total: 31.36G
[...]
\end{lstlisting}

\subsection{Rebalance mode}
After making any change to the cluster, it will automatically enter into
a rebalance mode. The rebalance process makes the data properly distributed
among the nodes according to the new cluster scheme. During the rebalance
all data operations on volumes can be performed as usual, but no changes
to the cluster itself are accepted. When the cluster is rebalancing, it
reports its new configuration in the status output under
\emph{``Target configuration''}. \textbf{IMPORTANT: } All nodes should be
online during the rebalance process. It is not possible to cancel a running
rebalance nor make any changes to the cluster during the process.

\subsection{Cluster resize}
The first modification we will perform is a global cluster resize.
\path{sxadm cluster --resize} provides an easy way to shrink or grow
the entire cluster, with changes applied to all nodes proportionally to
their current capacity in the cluster. In our cluster all four nodes
have equal sizes, therefore growing the cluster by 400GB, should result
in each node being resized by 100GB:
\begin{lstlisting}
$ sxadm cluster --resize +400G @cluster2
$ sxadm cluster --info @cluster2
Cluster UUID: cc8ab859-619e-4806-ade6-c32ab2db1665
Operating mode: read-write
Target configuration: 644245094400/192.168.100.1/d3f8ad83-d003-4aaa-bbfb-73359af85991 644245094400/192.168.100.2/abc2ed51-b4a8-46b6-a8ac-0beb58e697d2 644245094400/192.168.100.3/a343b7f9-0bef-4f03-8c6f-526ca12d75a9 644245094400/192.168.100.4/b9b05fc7-7a4b-417d-853b-ac56ed32f5d3 
Current configuration: 536870912000/192.168.100.1/d3f8ad83-d003-4aaa-bbfb-73359af85991 536870912000/192.168.100.2/abc2ed51-b4a8-46b6-a8ac-0beb58e697d2 536870912000/192.168.100.3/a343b7f9-0bef-4f03-8c6f-526ca12d75a9 536870912000/192.168.100.4/b9b05fc7-7a4b-417d-853b-ac56ed32f5d3
Operating mode: read-write
Distribution: 872eeecb-ebf9-4368-8150-beb23cd44edf(v.8) - checksum: 14098478712246199608
State of nodes:
  * node d3f8ad83-d003-4aaa-bbfb-73359af85991: addr: 192.168.100.1, capacity: 644245094400, status: follower, online: yes, activity: Relocation complete
  * node abc2ed51-b4a8-46b6-a8ac-0beb58e697d2: addr: 192.168.100.2, capacity: 644245094400, status: follower, online: yes, activity: Relocation complete
  * node a343b7f9-0bef-4f03-8c6f-526ca12d75a9: addr: 192.168.100.3, capacity: 644245094400, status: leader, online: yes, activity: Relocation complete
  * node b9b05fc7-7a4b-417d-853b-ac56ed32f5d3: addr: 192.168.100.4, capacity: 644245094400, status: follower, online: yes, activity: Relocation complete
\end{lstlisting}
All nodes were properly resized. When the rebalance process finishes, \emph{``Target
configuration''} will become \emph{``Current configuration''}.

\subsection{Node resize}
In order to modify a single node, we will use a generic option
\path{cluster --modify}, which takes a new configuration of the cluster.
First, we obtain the current configuration:
\begin{lstlisting}
$ sxadm cluster --info @cluster2
Cluster UUID: cc8ab859-619e-4806-ade6-c32ab2db1665
Operating mode: read-write
Current configuration: 644245094400/192.168.100.1/d3f8ad83-d003-4aaa-bbfb-73359af85991 644245094400/192.168.100.2/abc2ed51-b4a8-46b6-a8ac-0beb58e697d2 644245094400/192.168.100.3/a343b7f9-0bef-4f03-8c6f-526ca12d75a9 644245094400/192.168.100.4/b9b05fc7-7a4b-417d-853b-ac56ed32f5d3 
Operating mode: read-write
Distribution: 872eeecb-ebf9-4368-8150-beb23cd44edf(v.9) - checksum: 18024963750773516843
State of nodes:
  * node d3f8ad83-d003-4aaa-bbfb-73359af85991: addr: 192.168.100.1, capacity: 644245094400, status: follower, online: yes
  * node abc2ed51-b4a8-46b6-a8ac-0beb58e697d2: addr: 192.168.100.2, capacity: 644245094400, status: follower, online: yes
  * node a343b7f9-0bef-4f03-8c6f-526ca12d75a9: addr: 192.168.100.3, capacity: 644245094400, status: leader, online: yes
  * node b9b05fc7-7a4b-417d-853b-ac56ed32f5d3: addr: 192.168.100.4, capacity: 644245094400, status: follower, online: yes
\end{lstlisting}
In order to change the size of the node 192.168.100.1 to 700GB, we
provide a new configuration of the cluster with an updated specification of
that node that includes the new size and the other values left untouched:
\begin{lstlisting}
$ sxadm cluster --modify ^\marked{751619276800}^/192.168.100.1/d3f8ad83-d003-4aaa-bbfb-73359af85991 644245094400/192.168.100.2/abc2ed51-b4a8-46b6-a8ac-0beb58e697d2 644245094400/192.168.100.3/a343b7f9-0bef-4f03-8c6f-526ca12d75a9 644245094400/192.168.100.4/b9b05fc7-7a4b-417d-853b-ac56ed32f5d3 @cluster2
\end{lstlisting}
It's very important to provide proper node UUIDs, otherwise the cluster won't
be able to recognize the node changes. When the rebalance finishes, the new
configuration of the cluster is:
\begin{lstlisting}
$ sxadm cluster --info @cluster2
Cluster UUID: cc8ab859-619e-4806-ade6-c32ab2db1665
Operating mode: read-write
Current configuration: 751619276800/192.168.100.1/d3f8ad83-d003-4aaa-bbfb-73359af85991 644245094400/192.168.100.2/abc2ed51-b4a8-46b6-a8ac-0beb58e697d2 644245094400/192.168.100.3/a343b7f9-0bef-4f03-8c6f-526ca12d75a9 644245094400/192.168.100.4/b9b05fc7-7a4b-417d-853b-ac56ed32f5d3 
Operating mode: read-write
Distribution: 872eeecb-ebf9-4368-8150-beb23cd44edf(v.10) - checksum: 18024964785860635179
State of nodes:
  * node d3f8ad83-d003-4aaa-bbfb-73359af85991: addr: 192.168.100.1, capacity: 751619276800, status: follower, online: yes
  * node abc2ed51-b4a8-46b6-a8ac-0beb58e697d2: addr: 192.168.100.2, capacity: 644245094400, status: follower, online: yes
  * node a343b7f9-0bef-4f03-8c6f-526ca12d75a9: addr: 192.168.100.3, capacity: 644245094400, status: leader, online: yes
  * node b9b05fc7-7a4b-417d-853b-ac56ed32f5d3: addr: 192.168.100.4, capacity: 644245094400, status: follower, online: yes
\end{lstlisting}

\subsection{Node removal}
To easily remove a single node from the cluster, log into the node
and run \verb+sxsetup --deactivate+:
\begin{lstlisting}
# sxsetup --deactivate
This option will relocate all data stored on this node to other nodes in
the cluster and deactivate the node. The procedure can take more time
depending on the data size and the network speed. Please do not interrupt
it and don't turn off the node until the operation is finished.
Do you want to continue? (y/N) ^\marked{y}^
Waiting for cluster to finish data relocation...
Sending SIGTERM to 32644
Sending SIGTERM to 32654
Waiting for  32644 32654
The node has been successfully deactivated.
\end{lstlisting}

A generic approach to remove one or mode nodes at the same time, requires
removing node specifications from the current cluster configuration.
In order to remove the node 192.168.100.4 from the cluster discussed in previous
sections, provide a new cluster configuration \textbf{without} specification
of the node 192.168.100.4:
\begin{lstlisting}
$ sxadm cluster --modify 751619276800/192.168.100.1/d3f8ad83-d003-4aaa-bbfb-73359af85991 644245094400/192.168.100.2/abc2ed51-b4a8-46b6-a8ac-0beb58e697d2 644245094400/192.168.100.3/a343b7f9-0bef-4f03-8c6f-526ca12d75a9 @cluster2
$ sxadm cluster --info @cluster2
Cluster UUID: cc8ab859-619e-4806-ade6-c32ab2db1665
Operating mode: read-write
Target configuration: 751619276800/192.168.100.1/d3f8ad83-d003-4aaa-bbfb-73359af85991 644245094400/192.168.100.2/abc2ed51-b4a8-46b6-a8ac-0beb58e697d2 644245094400/192.168.100.3/a343b7f9-0bef-4f03-8c6f-526ca12d75a9 
Current configuration: 751619276800/192.168.100.1/d3f8ad83-d003-4aaa-bbfb-73359af85991 644245094400/192.168.100.2/abc2ed51-b4a8-46b6-a8ac-0beb58e697d2 644245094400/192.168.100.3/a343b7f9-0bef-4f03-8c6f-526ca12d75a9 644245094400/192.168.100.4/b9b05fc7-7a4b-417d-853b-ac56ed32f5d3 
Distribution: 872eeecb-ebf9-4368-8150-beb23cd44edf(v.11) - checksum: 16329829800547562843
State of nodes:
  * node d3f8ad83-d003-4aaa-bbfb-73359af85991: addr: 192.168.100.1, capacity: 751619276800, status: follower, online: yes, activity: Relocating data (1510 out of ~50733 blocks processed)
  * node abc2ed51-b4a8-46b6-a8ac-0beb58e697d2: addr: 192.168.100.2, capacity: 644245094400, status: follower, online: yes, activity: Relocating data (2217 out of ~48215 blocks processed)
  * node a343b7f9-0bef-4f03-8c6f-526ca12d75a9: addr: 192.168.100.3, capacity: 644245094400, status: leader, online: yes, activity: Relocating data (2053 out of ~49712 blocks processed)
  * node b9b05fc7-7a4b-417d-853b-ac56ed32f5d3: addr: 192.168.100.4, capacity: 644245094400, status: ^\marked{leaving}^, online: yes, activity: Relocating data (3696 out of ~38212 blocks processed)
\end{lstlisting}
The rebalance process will move all the data out of the node 192.168.100.4 and
deactivate it. When the node disappears from \path{cluster --info} output, it's
no longer part of the cluster and can be disabled physically. Using the above
approach it is possible to remove more nodes at the same time, by dropping their
specifications from the cluster configuration.

\subsection{Creating a bare node}
A bare node is a node, which is prepared to join a specific cluster, but is
not a part of the cluster yet. Bare nodes can be configured in order to
replace existing nodes or to join multiple nodes at once to the cluster,
rather than doing that one by one. A bare node can be configured in an
automatic way, similarly to the process described in
{\ifpdf\fref{sec:configfile}\else\ref{sec:configfile}\fi}
--- the only difference is that the option \path{--bare} must be additionally
passed to \path{sxsetup}. It can also be configured in interactive mode,
similarly to adding a new node as described in
{\ifpdf\fref{sec:addnode}\else\ref{sec:addnode}\fi}, by
running \path{sxsetup --bare} and answering the questions.
\begin{lstlisting}
# sxsetup --bare
[...]
SX node started successfully
Bare node created. Use 'sxadm cluster --modify' to join it to the cluster
or perform another operation.
Node specification: ^\marked{500G/192.168.100.5}^
\end{lstlisting}
When the setup is finished, it provides a node specification string, which
can be used with cluster modification options. Please notice the bare node
has no UUID assigned --- it will get it when joining the target cluster.

\subsection{Performing multiple changes at once}
Adding new nodes with \path{sxsetup} is a serialized process --- one node is
joined to a cluster --- a rebalance is triggered and then another node can be
added. With \path{sxadm cluster --modify} multiple operations can be merged
and performed at once, resulting in a single and shorter data rebalance
process. In the following example, we will replace a couple of nodes in the
cluster, by adding two larger nodes and removing two existing smaller nodes.
First, we obtain the current cluster configuration:
\begin{lstlisting}
$ sxadm cluster --info @cluster2
Cluster UUID: cc8ab859-619e-4806-ade6-c32ab2db1665
Operating mode: read-write
Current configuration: ^\marked{536870912000/192.168.100.1/d3f8ad83-d003-4aaa-bbfb-73359af85991}^ ^\marked{536870912000/192.168.100.2/abc2ed51-b4a8-46b6-a8ac-0beb58e697d2}^ ^\marked{536870912000/192.168.100.3/a343b7f9-0bef-4f03-8c6f-526ca12d75a9}^ ^\marked{536870912000/192.168.100.4/b9b05fc7-7a4b-417d-853b-ac56ed32f5d3}^
Distribution: 872eeecb-ebf9-4368-8150-beb23cd44edf(v.11) - checksum: 16116260632263325108
State of nodes:
  * node d3f8ad83-d003-4aaa-bbfb-73359af85991: addr: 192.168.100.1, capacity: 536870912000, status: follower, online: yes
  * node abc2ed51-b4a8-46b6-a8ac-0beb58e697d2: addr: 192.168.100.2, capacity: 536870912000, status: follower, online: yes
  * node a343b7f9-0bef-4f03-8c6f-526ca12d75a9: addr: 192.168.100.3, capacity: 536870912000, status: leader, online: yes
  * node b9b05fc7-7a4b-417d-853b-ac56ed32f5d3: addr: 192.168.100.4, capacity: 536870912000, status: follower, online: yes
\end{lstlisting}
It tells us there are four 500GB nodes. Now we create a couple of bare nodes: 192.168.100.5
and 192.168.100.6, both 1TB in size:
\begin{lstlisting}
-- on node 192.168.100.5 --
# sxsetup --bare
[...]
SX node started successfully
Bare node created. Use 'sxadm cluster --modify' to join it to the cluster
or perform another operation.
Node specification: ^\markedG{1T/192.168.100.5}^
\end{lstlisting}
\begin{lstlisting}
-- on node 192.168.100.6 --
# sxsetup --bare
[...]
SX node started successfully
Bare node created. Use 'sxadm cluster --modify' to join it to the cluster
or perform another operation.
Node specification: ^\markedG{1T/192.168.100.6}^
\end{lstlisting}
With the following command, we will remove nodes 192.168.100.3 and
192.168.100.4 and add a couple of larger nodes 192.168.100.5 and
192.167.100.6. In order to do that, we provide a new cluster configuration,
consisting of the current specifications for nodes 192.168.100.1 and
192.168.100.2 as well as the bare nodes:
\begin{lstlisting}
$ sxadm cluster --modify ^\marked{536870912000/192.168.100.1/d3f8ad83-d003-4aaa-bbfb-73359af85991}^ ^\marked{536870912000/192.168.100.2/abc2ed51-b4a8-46b6-a8ac-0beb58e697d2}^ ^\markedG{1T/192.168.100.5}^ ^\markedG{1T/192.168.100.6}^ @cluster2
\end{lstlisting}
After issuing the command, the rebalance process is started, which moves all
data from the nodes 192.168.100.3 and 192.168.100.4 and balances the data across
the cluster, which now also includes the 1TB nodes:
\begin{lstlisting}
$ sxadm cluster --info @cluster2
Cluster UUID: cc8ab859-619e-4806-ade6-c32ab2db1665
Operating mode: read-write
Target configuration: ^\marked{536870912000/192.168.100.1/d3f8ad83-d003-4aaa-bbfb-73359af85991}^ ^\marked{536870912000/192.168.100.2/abc2ed51-b4a8-46b6-a8ac-0beb58e697d2}^ ^\markedG{1099511627776/192.168.100.5/42ea1ec2-4127-491a-9ff9-d9fdfd7c92d0}^ ^\markedG{1099511627776/192.168.100.6/5f26e559-fca0-44aa-b2d6-eb6e8e1156b1}^
Current configuration: 536870912000/192.168.100.1/d3f8ad83-d003-4aaa-bbfb-73359af85991 536870912000/192.168.100.2/abc2ed51-b4a8-46b6-a8ac-0beb58e697d2 536870912000/192.168.100.3/a343b7f9-0bef-4f03-8c6f-526ca12d75a9 536870912000/192.168.100.4/b9b05fc7-7a4b-417d-853b-ac56ed32f5d3
Distribution: 872eeecb-ebf9-4368-8150-beb23cd44edf(v.12) - checksum: 16116260632263325108
State of nodes:
  * node d3f8ad83-d003-4aaa-bbfb-73359af85991: addr: 192.168.100.1, capacity: 536870912000, status: follower, online: yes, activity: Relocating data (1510 out of ~37733 blocks processed)
  * node abc2ed51-b4a8-46b6-a8ac-0beb58e697d2: addr: 192.168.100.2, capacity: 536870912000, status: follower, online: yes, activity: Relocating data (2217 out of ~38215 blocks processed)
  * node a343b7f9-0bef-4f03-8c6f-526ca12d75a9: addr: 192.168.100.3, capacity: 536870912000, status: ^\marked{leaving}^, online: yes, activity: Relocating data (2053 out of ~39712 blocks processed)
  * node b9b05fc7-7a4b-417d-853b-ac56ed32f5d3: addr: 192.168.100.4, capacity: 536870912000, status: ^\marked{leaving}^, online: yes, activity: Relocating data (3696 out of ~38212 blocks processed)
  * node 42ea1ec2-4127-491a-9ff9-d9fdfd7c92d0: addr: 192.168.100.5, capacity 1099511627776, status: ^\marked{joining}^, online: yes, activity: Relocating data (2976 out of ~43398 blocks processed)
  * node 5f26e559-fca0-44aa-b2d6-eb6e8e1156b1: addr: 192.168.100.6, capacity 1099511627776, status: ^\marked{joining}^, online: yes, activity: Relocating data (3153 out of ~43343 blocks processed)
\end{lstlisting}
When the rebalance finishes, the cluster consists of two 500GB nodes:
192.168.100.1 and 192.168.100.2 and two 1TB nodes: 192.168.100.5 and
192.168.100.6:
\begin{lstlisting}
$ sxadm cluster --info @cluster2
Cluster UUID: cc8ab859-619e-4806-ade6-c32ab2db1665
Operating mode: read-write
Current configuration: 536870912000/192.168.100.1/d3f8ad83-d003-4aaa-bbfb-73359af85991 536870912000/192.168.100.2/abc2ed51-b4a8-46b6-a8ac-0beb58e697d2 1099511627776/192.168.100.5/42ea1ec2-4127-491a-9ff9-d9fdfd7c92d0 1099511627776/192.168.100.6/5f26e559-fca0-44aa-b2d6-eb6e8e1156b1
Distribution: 872eeecb-ebf9-4368-8150-beb23cd44edf(v.12) - checksum: 16116260632263325108
State of nodes:
  * node d3f8ad83-d003-4aaa-bbfb-73359af85991: addr: 192.168.100.1, capacity: 751619276800, status: follower, online: yes
  * node abc2ed51-b4a8-46b6-a8ac-0beb58e697d2: addr: 192.168.100.2, capacity: 644245094400, status: follower, online: yes
  * node 42ea1ec2-4127-491a-9ff9-d9fdfd7c92d0: addr: 192.168.100.5, capacity 1099511627776, status: leader, online: yes
  * node 5f26e559-fca0-44aa-b2d6-eb6e8e1156b1: addr: 192.168.100.6, capacity 1099511627776, status: follower, online: yes
\end{lstlisting}
The nodes 192.168.100.3 and 192.168.100.4 are no longer part of the cluster
and can be turned off.

\section{Zone configuration}
By default, replication is performed across individual nodes. \SX provides
a mechanism to group the nodes into zones, which can be enabled when the
cluster should be rack aware or its data distributed across different regions
(geo-replica).
\begin{lstlisting}
$ sxadm cluster --info @cluster2
Cluster UUID: cc8ab859-619e-4806-ade6-c32ab2db1665
Operating mode: read-write
Current configuration: 644245094400/192.168.100.1/d3f8ad83-d003-4aaa-bbfb-73359af85991 644245094400/192.168.100.2/abc2ed51-b4a8-46b6-a8ac-0beb58e697d2 644245094400/192.168.100.3/a343b7f9-0bef-4f03-8c6f-526ca12d75a9 644245094400/192.168.100.4/b9b05fc7-7a4b-417d-853b-ac56ed32f5d3 
Operating mode: read-write
Distribution: 872eeecb-ebf9-4368-8150-beb23cd44edf(v.9) - checksum: 18024963750773516843
State of nodes:
  * node d3f8ad83-d003-4aaa-bbfb-73359af85991: addr: 192.168.100.1, capacity: 644245094400, status: follower, online: yes
  * node abc2ed51-b4a8-46b6-a8ac-0beb58e697d2: addr: 192.168.100.2, capacity: 644245094400, status: follower, online: yes
  * node a343b7f9-0bef-4f03-8c6f-526ca12d75a9: addr: 192.168.100.3, capacity: 644245094400, status: leader, online: yes
  * node b9b05fc7-7a4b-417d-853b-ac56ed32f5d3: addr: 192.168.100.4, capacity: 644245094400, status: follower, online: yes
\end{lstlisting}
The test cluster has 4 nodes and allows a maximum of 4 replicas. With the
replica 2, all data gets replicated twice on 2 different nodes of the
cluster. If the nodes are located in two different racks or datacenters,
it might happen the replica is made on two nodes in the same location.
One can ensure the data gets distributed across different locations, by
grouping the zones into nodes. The zone configuration is passed as the
last argument to the already discussed \path{cluster --modify} option,
which takes a new configuration of the cluster. The format of the zone entry is:
\begin{lstlisting}
ZoneName1:UUID1,UUID2,...;ZoneName2:UUID3,UUID4,...
\end{lstlisting}
In the following example, we will group nodes 192.168.100.1 and
192.168.100.2 into zone "Rack1", and nodes 192.168.100.3 and
192.168.100.4 into "Rack2". To do this, we call \path{cluster --modify}
with the current configuration and append the zone configuration after
the node list:
\ifpdf
\begin{lstlisting}
$ sxadm cluster --modify 644245094400/192.168.100.1/d3f8ad83-d003-4aaa-bbfb-73359af85991 644245094400/192.168.100.2/abc2ed51-b4a8-46b6-a8ac-0beb58e697d2 644245094400/192.168.100.3/a343b7f9-0bef-4f03-8c6f-526ca12d75a9 644245094400/192.168.100.4/b9b05fc7-7a4b-417d-853b-ac56ed32f5d3 ^\marked{"Rack1:d3f8ad83-d003-4aaa-bbfb-73359af85991,abc2ed51-b4a8-46b6-a8ac-0beb58e697d2;}^ ^\marked{Rack2:a343b7f9-0bef-4f03-8c6f-526ca12d75a9,b9b05fc7-7a4b-417d-853b-ac56ed32f5d3"}^ @cluster2
\end{lstlisting}
\else
\begin{lstlisting}
$ sxadm cluster --modify 644245094400/192.168.100.1/d3f8ad83-d003-4aaa-bbfb-73359af85991 644245094400/192.168.100.2/abc2ed51-b4a8-46b6-a8ac-0beb58e697d2 644245094400/192.168.100.3/a343b7f9-0bef-4f03-8c6f-526ca12d75a9 644245094400/192.168.100.4/b9b05fc7-7a4b-417d-853b-ac56ed32f5d3 ^\marked{"Rack1:d3f8ad83-d003-4aaa-bbfb-73359af85991,abc2ed51-b4a8-46b6-a8ac-0beb58e697d2;Rack2:a343b7f9-0bef-4f03-8c6f-526ca12d75a9,b9b05fc7-7a4b-417d-853b-ac56ed32f5d3"}^ @cluster2
\end{lstlisting}
\fi
After changing the zone configuration, the cluster needs to relocate the data
according to the new distribution rules. With the new cluster configuration,
the maximum replica is now 2 (the zone configuration would fail to apply, if
the cluster contained volumes with replica higher than 2), and the data will
be replicated across two zones. In a zone-enabled cluster, the maximum replica
always equals to the number of zones plus the number of nodes, which are not
part of any zone. When the data relocation is complete, the cluster will report
the zone for each node:
\begin{lstlisting}
$ sxadm cluster --info @cluster2
Cluster UUID: cc8ab859-619e-4806-ade6-c32ab2db1665
Operating mode: read-write
Current configuration: 644245094400/192.168.100.1/d3f8ad83-d003-4aaa-bbfb-73359af85991 644245094400/192.168.100.2/abc2ed51-b4a8-46b6-a8ac-0beb58e697d2 644245094400/192.168.100.3/a343b7f9-0bef-4f03-8c6f-526ca12d75a9 644245094400/192.168.100.4/b9b05fc7-7a4b-417d-853b-ac56ed32f5d3 Rack1:d3f8ad83-d003-4aaa-bbfb-73359af85991,abc2ed51-b4a8-46b6-a8ac-0beb58e697d2;Rack2:a343b7f9-0bef-4f03-8c6f-526ca12d75a9,b9b05fc7-7a4b-417d-853b-ac56ed32f5d3
Operating mode: read-write
Distribution: 872eeecb-ebf9-4368-8150-beb23cd44edf(v.10) - checksum: 18024963750773516843
State of nodes:
  * node d3f8ad83-d003-4aaa-bbfb-73359af85991: addr: 192.168.100.1, capacity: 644245094400, zone: ^\marked{Rack1}^, status: follower, online: yes
  * node abc2ed51-b4a8-46b6-a8ac-0beb58e697d2: addr: 192.168.100.2, capacity: 644245094400, zone: ^\marked{Rack1}^, status: follower, online: yes
  * node a343b7f9-0bef-4f03-8c6f-526ca12d75a9: addr: 192.168.100.3, capacity: 644245094400, zone: ^\marked{Rack2}^, status: leader, online: yes
  * node b9b05fc7-7a4b-417d-853b-ac56ed32f5d3: addr: 192.168.100.4, capacity: 644245094400, zone: ^\marked{Rack2}^, status: follower, online: yes
\end{lstlisting}
With this cluster configuration, volumes with replica 2 will be always fully
replicated in both zones Rack1 and Rack2 (the data will be load-balanced across
the nodes within each zone).

\section{Cluster backup and restore}
This section describes how to backup and restore the entire cluster.

\subsection{Backup}
The cluster backup can be automated with the \path{sxdump} tool, available
from \url{https://pypi.python.org/pypi/sxdump/}. After installing \path{sxdump}
(manually or automatically with \path{pip}), run the following command:
\begin{lstlisting}
# sxdump --backup-dir /var/backups/sx/ sx://admin@mycluster
Generating sx-backup.sh and sx-restore.sh in the current directory
Review sx-backup.sh (uncomment/edit paths as necessary)
Run sx-backup.sh on the old cluster
Review sx-restore.sh (uncomment/edit paths as necessary)
Stop old cluster
Use sxsetup --config-file to setup the new cluster
Run sx-restore.sh on the new cluster
\end{lstlisting}
It creates two shell scripts in the current directory: \verb+sx-backup.sh+ and
\verb+sx-restore.sh+. The first script will backup all data from the cluster,
while the other contains information on how to recreate the cluster structure,
including all volumes, users, ACLs, and settings. Running
\verb+sx-backup.sh+ will create a copy of all files in the cluster:
\begin{lstlisting}
# ./sx-backup.sh

Backing up volume sx://admin@sxtest/vol1

Downloading /video.mkv (size: 1.22GB)
Transferred 1.22GB in 8s (@154.35MB/s)           
[...]
\end{lstlisting}
When the script finishes, the data from the cluster will be backed up
in \verb+/var/backups/sx+. \textbf{No data from encrypted volumes will be
backed up --- those have to be processed manually.}

\subsection{Restore}
In order to restore the cluster, including all volumes, users, and ACLs run
\verb+sx-restore.sh+ created by \verb+sxdump+ against a new cluster. You may
need to edit the file in case the cluster name or location of the backup changed.

\section{Cluster healing}
It may happen one or more nodes are permanently lost due to external causes.
When that happens, some operations might only be possible in read-only mode,
until the broken nodes get replaced or removed from the cluster.
\SX can automatically detect offline or broken nodes and automatically disable
them. It uses the Raft algorithm, which achieves consensus via an elected leader.
The leader checks if it has received heartbeats from all nodes in the cluster, and
if some nodes didn't respond for a specified amount of time, they can be
blacklisted. Due to the design of the Raft algorithm and the required majority,
the automatic healing will only work for clusters with 3 or more nodes. For
smaller clusters the operation has to be performed manually, as described below.

\subsection{Configuring auto-healing}
The following cluster settings are used to fine-tune the heartbeat and auto-healing
process:
\begin{itemize}
    \item \textbf{hb\_keepalive}\\
    This option sets the interval between hearbeats (in seconds). The default
    value is 20 seconds.
    \item \textbf{hb\_warntime}\\
    The hb\_warntime option is used to specify how quickly heartbeat should issue
    a warning, that a node is unreachable. The default setting is 120 seconds.
    \item \textbf{hb\_initdead}\\
    The option is used to set the time that it takes to declare a node dead when
    heartbeat is first started. It helps to avoid false reports in case a node
    or its operating system takes more time to start proper network operations.
    The default value for this option is 120 seconds.
    \item \textbf{hb\_deadtime}\\
    This option sets the time, after which an unreachable node is considered dead
    and gets marked as broken. A broken node should be later replaced, as described
    below. This option is turned off by default.
\end{itemize}
You can obtain the current value of any setting by running the following command:
\begin{lstlisting}
$ sxadm cluster --get-param=hb_keepalive @cluster2
hb_keepalive=20
\end{lstlisting}
To automatically mark broken nodes and disable them from regular cluster
operations after 10 minutes of downtime, set the following options:
\begin{lstlisting}
$ sxadm cluster --set-param="hb_initdead=600" @cluster2
$ sxadm cluster --set-param="hb_deadtime=600" @cluster2
hb_keepalive=20
\end{lstlisting}
Now, when the cluster detects a node, which didn't perform a valid heartbeat
for more than 10 minutes, it will be automatically set as faulty and the cluster
will allow write operations again. In the example below, the node 192.168.100.4
has been set as faulty:
\begin{lstlisting}
$ sxadm cluster --info @cluster2
Cluster UUID: cc8ab859-619e-4806-ade6-c32ab2db1665
Operating mode: read-write
Current configuration: 644245094400/192.168.100.1/d3f8ad83-d003-4aaa-bbfb-73359af85991 644245094400/192.168.100.2/abc2ed51-b4a8-46b6-a8ac-0beb58e697d2 644245094400/192.168.100.3/a343b7f9-0bef-4f03-8c6f-526ca12d75a9 644245094400/192.168.100.4/b9b05fc7-7a4b-417d-853b-ac56ed32f5d3 
Distribution: 872eeecb-ebf9-4368-8150-beb23cd44edf(v.9) - checksum: 18024963750773516843
State of nodes:
  * node d3f8ad83-d003-4aaa-bbfb-73359af85991: addr: 192.168.100.1, capacity: 644245094400, status: follower, online: yes
  * node abc2ed51-b4a8-46b6-a8ac-0beb58e697d2: addr: 192.168.100.2, capacity: 644245094400, status: follower, online: yes
  * node a343b7f9-0bef-4f03-8c6f-526ca12d75a9: addr: 192.168.100.3, capacity: 644245094400, status: leader, online: yes
  * node b9b05fc7-7a4b-417d-853b-ac56ed32f5d3: addr: 192.168.100.4, capacity: 644245094400, status: ^\marked{**~FAULTY~**}^, online: ^\marked{**~NO~**}^
\end{lstlisting}
The faulty nodes should be later replaced as described below.

\subsection{Marking broken nodes manually}
If the automatic detection and marking of broken nodes is not active (eg. when
there's less than 3 nodes in the cluster and the Raft algorithm cannot be
activated), the bad nodes can still be marked manually. Run the following
command and provide the full specification of the broken node to mark it
faulty:
\begin{lstlisting}
$ sxadm cluster --set-faulty ^\marked{644245094400/192.168.100.4/b9b05fc7-7a4b-417d-853b-ac56ed32f5d3}^ @cluster2
\end{lstlisting}
Cluster with faulty nodes, which have been marked, will properly operate in
read-write mode, however no changes to the cluster structure will be allowed
until the faulty nodes get properly replaced as described in the next
subsection.

\subsection{Replacing broken nodes}
Skylable \SX provides an option to automatically rebuild
a lost node and gather as much data as possible from other nodes.
\textbf{Please never use this method against properly working nodes:} it
assumes the node's data is permanently lost and can only retrieve missing data
for volumes with replica higher than 1 --- healthy nodes can be replaced using
\path{--modify} option as described in the previous section. In the following
example, we assume the node 192.168.100.4 is no longer available (just lost
or already marked as faulty) and we will replace it with a new node 192.168.100.5.
First we need to prepare a bare node 192.168.100.5 of the exact size as the broken
node we are replacing, in this case it's 600GB:
\begin{lstlisting}
-- on node 192.168.100.5 --
# sxsetup --bare
[...]
SX node started successfully
Bare node created. Use 'sxadm cluster --modify' to join it to the cluster
or perform another operation.
Node specification: ^\markedG{600G/192.168.100.5}^
\end{lstlisting}
Now we issue the following command, which uses the specification (size and UUID)
of the broken node but points to the new IP address:
\begin{lstlisting}
$ sxadm cluster --replace-faulty
    ^\marked{644245094400}^/^\markedG{192.168.100.5}^/^\marked{b9b05fc7-7a4b-417d-853b-ac56ed32f5d3}^ @cluster2
\end{lstlisting}
The broken node is immediately replaced with the new one, and the healing process
is started:
\begin{lstlisting}
$ sxadm cluster --info @cluster2
Cluster UUID: cc8ab859-619e-4806-ade6-c32ab2db1665
Operating mode: read-write
Current configuration: 644245094400/192.168.100.1/d3f8ad83-d003-4aaa-bbfb-73359af85991 644245094400/192.168.100.2/abc2ed51-b4a8-46b6-a8ac-0beb58e697d2 644245094400/192.168.100.3/a343b7f9-0bef-4f03-8c6f-526ca12d75a9 644245094400/192.168.100.5/b9b05fc7-7a4b-417d-853b-ac56ed32f5d3 
Distribution: 872eeecb-ebf9-4368-8150-beb23cd44edf(v.10) - checksum: 18024963750773516843
State of nodes:
  * node d3f8ad83-d003-4aaa-bbfb-73359af85991: addr: 192.168.100.1, capacity: 644245094400, status: follower, online: yes
  * node abc2ed51-b4a8-46b6-a8ac-0beb58e697d2: addr: 192.168.100.2, capacity: 644245094400, status: follower, online: yes
  * node a343b7f9-0bef-4f03-8c6f-526ca12d75a9: addr: 192.168.100.3, capacity: 644245094400, status: leader, online: yes
  * node b9b05fc7-7a4b-417d-853b-ac56ed32f5d3: addr: 192.168.100.5, capacity: 644245094400, status: follower, online: yes, activity: ^\marked{Healing}^ ^\marked{blocks}^
\end{lstlisting}
During the repair process client operations should be back to normal. The same
steps can be used to replace a broken node without changing its IP address, in
that case the bare node must be prepared and available with the IP address of
the broken one. It is also possible to repair more than one node at a time by
passing more node specifications to \path{--replace-faulty}.

\subsection{Setting read-only mode}
The cluster can be set into read-only mode to perform maintainance or
temporarily stop clients from uploading new data by running the following
command:
\begin{lstlisting}
$ sxadm cluster --set-mode=ro @cluster2
Successfully switched cluster to read-only mode
\end{lstlisting}
To switch the cluster back to read-write mode, run the same command with
\verb+rw+ argument:
\begin{lstlisting}
$ sxadm cluster --set-mode=rw @cluster2
Successfully switched cluster to read-write mode
\end{lstlisting}

\subsection{Checking storage integrity}
\SX provides a tool that performs a deep check of the storage structure
and verifies if the data on disk, stored in a special format called \textsl{HashFS},
is not corrupted. The tool will calculate and compare checksums of all
data blocks and report any inconsistencies. In order to perform this check,
the cluster needs to be set to read-only mode or a particular node needs to
be temporarily turned off (with \verb+sxserver stop+). When the cluster is
in read-only mode, you can perform the check on all nodes at the same time.
Run the following command on the node you want to check (run
\path{sxsetup --info} if you don't remember the location of the storage):
\begin{lstlisting}
# sxadm node --check /var/lib/sxserver/storage/
[sx_hashfs_check]: Integrity check started   
HashFS is clean, no errors found
\end{lstlisting}

\subsection{Compacting node data}
The local storage of \SX consists of special databases and data files, which
store the information about files and cluster configuration as well as the
actual blocks of data. When a file gets deleted, all blocks belonging
to that file, which are not shared by other files are marked as free and
can be reused. \SX will only allocate more space on disk, if it cannot reuse
existing blocks within data files. It will, however, not return the free blocks
to the system automatically. If there's a need to free some disk space, the
local storage can be easily compacted:
\begin{lstlisting}
# sxsetup --compact
The node will be stopped now and started again when compacting is finished
Sending SIGTERM to 28310
Sending SIGTERM to 28320
Waiting for  28310 28320
Operation complete (disk space freed: 1.95G)
Starting SX.fcgi
Starting sxhttpd
SX node started successfully
\end{lstlisting}
In case compacting doesn't free any disk space, force the garbage collector
to mark blocks of recently removed files as free and then try again. The
garbage collector is run with the following command:
\begin{lstlisting}
$ sxadm cluster --force-gc @cluster2
\end{lstlisting}

\subsection{Data recovery}
It is possible to recover local data in case a node gets damaged. Please
perform the following command and \path{sxadm} will try to extract as
much data as possible from the local storage:
\begin{lstlisting}
# sxadm node --extract=/tmp/RECOVERED /var/lib/sxserver/storage/
Finished data extraction from node /var/lib/sxserver/storage/
\end{lstlisting}

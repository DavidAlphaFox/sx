\chapter{Advanced}
This chapter describes some advanced details of the \SX design.

\section{Data distribution}

\ifpdf
\begin{figure}
	\centering
	% !TeX root = sxmanual.tex
\colorlet{c0}{yellow!20}
\colorlet{c1}{green!10}
\colorlet{c2}{green!30}

\begin{tikzpicture}[auto, thick, >=stealth', shorten >=1pt,
array/.style={rectangle split,rectangle split horizontal, rectangle split parts=#1,draw, minimum height=4em,
inner sep=1em
}]
\tikzstyle{sx}=[draw,circle,black,bottom color=skylableblue, node distance=6em,
top color= white, text=black,minimum width=2em]

\node[sx] (sx1) {\SX node};
\node[sx] (sx2) [right of=sx1] {\SX node};
\node[sx] (sx3) [right of=sx2] {\SX node};
\node[sx] (sx4) [right of=sx3] {\SX node};

\node[] (sx)  at ($(sx1)!0.5!(sx4)$) {};
\draw[] node[draw,dashed,cloud,cloud puffs=15,cloud puff arc=120,inner ysep=1em,aspect=2,color=black,
minimum height=11em,minimum width=25em] at (sx) {}; 


\node[array=10] (f1) [below of=sx, node distance=15em,
rectangle split part fill={c0,c1,c2,c1,c2,c1,c2,c1,c0,c0}] {
	\nodepart{one}=
	\nodepart{two}2
	\nodepart{three}3
	\nodepart{four}4
	\nodepart{five}5
	\nodepart{six}6
	\nodepart{seven}7
	\nodepart{eight}8
	\nodepart{nine}=
	\nodepart{ten}=
};


\draw[*-*]
let \p1 = ($(f1.one)+(0,1em)$),
\p2 = ($(\p1)!0.55!(sx1)$),
\p3 = ($(sx1)+(0.5em,0)$) in
(\p1) |- (\x3,\y2);

\draw[*-]
let \p1 = ($(f1.four)+(0,1em)$),
\p2 = (sx) in
(\p1) |- ($(\p1)!0.55!(\x1,\y2)$);

\draw[*->]
let \p1 = ($(f1.four)+(0,1em)$),
\p2 = (sx) in
($(\p1)!0.55!(\x1,\y2)+(0.3em,0)$) -| (sx1);


\draw[*->]
	let \p1 = ($(f1.three)+(0,1em)$),
	\p2 = (sx) in
	(\p1) |- ($(\p1)!0.6!(\x1,\y2)$) -| (sx2);
	
\draw[*-*]
	let \p1 = ($(f1.two)+(0,1em)$),
	\p2 = ($(\p1)!0.5!(sx3)$),
	\p3 = ($(sx3)+(0.5em,0)$) in
	(\p1) |- (\x3,\y2);
	
\draw[*->]
	let \p1 = ($(f1.five)+(0,1em)$),
	\p2 = (sx) in
	(\p1) |- ($(\p1)!0.4!(\x1,\y2)$) -| (sx4);
	
\draw[*->]
	let \p1 = ($(f1.six)+(0,1em)$),
	\p2 = (sx) in
	(\p1) |- ($(\p1)!0.45!(\x1,\y2)$) -| (sx3);

\draw[*-]
let \p1 = ($(f1.seven)+(0,1em)$),
\p2 = (sx) in
(\p1) |- ($(\p1)!0.4!(\x1,\y2)$);

\draw[*->]
let \p1 = ($(f1.seven)+(0,1em)$),
\p2 = (sx) in
($(\p1)!0.4!(\x1,\y2)+(-0.3em,0)$) -| (sx4);


\draw[*->]
let \p1 = ($(f1.eight)+(0,1em)$),
\p2 = (sx) in
(\p1) |- ($(\p1)!0.55!(\x1,\y2)$) -| (sx1);	
	

%\draw[*->] ($(file1.one)+(0.25em,0)$)  -| (sx3);
%\draw[*->] ($(file1.four)+(0.25em,0)$)  -| (sx4);

\end{tikzpicture}

	\caption{Data distribution}
	\label{fig:dist}
\end{figure}
\fi

Files are divided into blocks of equal size, which depend on the file's size,
and distributed among \SX nodes using a consistent hashing algorithm. This
ensures that the storage among all nodes in the cluster is properly balanced,
and that when nodes are added or removed from the cluster only a minimal amount
of data gets moved.

The cluster exposes an HTTP(S) REST API designed around deduplicated storage:
equal blocks of data get stored only once. This has the advantage of reduced
bandwidth usage, better resume handling, and that the data is immutable (only
the metadata is mutable).

In the example shown in \ifpdf\fref{fig:dist}\else\ref{fig:dist}\fi there is a 
file divided into 10 blocks, where the first and last two blocks are
equal --- only the first block of them is uploaded. The rest of the
blocks are distributed among all nodes in the cluster. Inevitably some
nodes will receive more than one block, in which case the protocol
supports an efficient batched upload and download mode.

In this example each block of data was stored only once, what is called
\emph{replica 1}. If one of the nodes that stores a block of the file goes down,
the file will be unreachable. To make the data more resistant one should
use a replica count higher than 1, which means that the data will be
duplicated on multiple nodes. A volume with replica 2 can survive a loss of
at most 1 node, with replica 3 the loss of at most 2 nodes, etc. The maximum
replica count is limited by the number of nodes. High replica counts increase
reliability for downloads, at the cost of increased latency on uploads, and
lower fault-tolerance on uploads (all replica nodes must be up for uploads to
succeed).

\ifpdf
\begin{figure}
	\centering
	% !TeX root = sxmanual.tex
\colorlet{c0}{yellow!20}
\colorlet{c1}{green!10}
\colorlet{c2}{green!30}
\colorlet{fail}{red!80!black}
\colorlet{good}{green!80!black}

\begin{tikzpicture}[auto, thick, >=stealth', shorten >=1pt,
array/.style={rectangle split,rectangle split horizontal, rectangle split parts=#1,draw, minimum height=4em,
inner sep=1em
}]
\tikzstyle{sx}=[draw,circle,black,bottom color=skylableblue, node distance=6em,
top color= white, text=black,minimum width=2em]

\node[sx] (sx1) {\SX node};
\node[sx] (sx2) [right of=sx1] {\SX node};
\node[sx] (sx3) [right of=sx2] {\SX node};
\node[sx] (sx4) [right of=sx3] {\SX node};

\node[] (sx)  at ($(sx1)!0.5!(sx4)$) {};
\draw[] node[draw,dotted,cloud,cloud puffs=15,cloud puff arc=120,inner ysep=1em,aspect=2,color=black,
minimum height=20em,minimum width=25em] at (sx) {}; 


\node[array=10] (f1) [below of=sx, node distance=15em,
rectangle split part fill={c0,c1,c2,c1,c2,c1,c2,c1,c0,c0}] {
	\nodepart{one}=
	\nodepart{two}2
	\nodepart{three}3
	\nodepart{four}4
	\nodepart{five}5
	\nodepart{six}6
	\nodepart{seven}7
	\nodepart{eight}8
	\nodepart{nine}=
	\nodepart{ten}=
};


\draw[*->,fail,dashed]
let \p1 = ($(f1.one)+(-0.5em,1em)$),
\p2 = ($(\p1)!0.6!(sx1)$),
\p3 = ($(sx1)$) in
(\p1) |- (\x3,\y2) -| (sx1);

\path[->,bend left,fail,dashed]
(sx1.north) edge [bend left] (sx2.north);

\path[->,bend left,fail,dashed]
(sx1.north) edge [bend left] (sx4.north);

\draw[*->,dashdotted,very thick,fail]
let \p1 = ($(f1.one)+(0.25em,1em)$),
\p2 = ($(\p1)!0.5!(sx2)$),
\p3 = ($(sx2)$) in
(\p1) |- (\x3,\y2) -| (sx2);

\path[->,bend left, very thick, dashdotted,fail]
(sx2.south) edge [bend left] (sx1.south);

\path[->,bend left, very thick, dashdotted, fail]
(sx2.south) edge [bend right] (sx4.south);

\draw[*->,good]
let \p1 = ($(f1.one)+(1em,1em)$),
\p2 = ($(\p1)!0.4!(sx4)$),
\p3 = ($(sx4)$) in
(\p1) |- (\x3,\y2) -| (sx4);

\path[->,bend left, good]
(sx4.north) edge [bend right=60] (sx1.north);

\path[->,bend left, good]
(sx4.north) edge [bend right=60] (sx2.north);


\end{tikzpicture}
	\caption{Data replication}
	\label{fig:replication}
\end{figure}
\fi

The handling of replicas is illustrated in \fref{fig:replication} for
a replica 3 volume. The client tries to upload the data to a given node,
which is then responsible for replicating the data inside the cluster
asynchronously. If the client fails to upload to a specific node it can
retry on the next one, and so on. Each time the receiving node will
replicate the data inside the cluster.

\section{Global objects}
\newcommand{\col}[1]{#1}
\begin{table}
	\centering
	\begin{tabular}{lccccccccc}
		\hline
		\multirow{2}{*}{node} & \multicolumn{2}{c}{volA\_r1} & \multicolumn{2}{c}{volB\_r2} & \multicolumn{2}{c}{volC\_r3} & \multicolumn{3}{c}{users} \\
		& \col{ACL} & \col{files} & \col{ACL} & \col{files} & \col{ACL} & \col{files} & admin & u1 & u2 \\
		\hline
		node1 & \cmark & \cmark & \cmark & \xmark & \cmark &  \xmark  & \cmark & \cmark & \cmark \\
		node2 & \cmark & \xmark & \cmark &  \cmark & \cmark & \cmark & \cmark & \cmark & \cmark \\
		node3 & \cmark & \xmark & \cmark & \cmark & \cmark & \cmark & \cmark & \cmark & \cmark \\
		node4 & \cmark & \xmark & \cmark & \xmark & \cmark &  \cmark & \cmark & \cmark & \cmark \\ 
		\hline
	\end{tabular}
	\caption{Global objects}
	\label{tab:globals}
\end{table}

Users, volumes and privileges are stored globally --- on each node in the
cluster --- and changing them requires cooperation of all nodes.

\begin{description}
\item[user] each user is issued an authentication token. All requests are
    signed using HMAC and the authentication token.
\item[admin users] the privileged users can perform all administrative tasks
    such as volume and user management
\item[volumes] used to group several files, owned by a specific user. Each
    volume has a replica count and metadata associated with it.
\item[volume ACL] the volume owner by default has full access to the volume.
    The owner (and the cluster admin) can grant and revoke permissions for
    other users\footnote{It's impossible to revoke privileges for admin users,
    they always have full access to all volumes.}.
\end{description}

As shown in \ifpdf\tref{tab:globals}\else\ref{tab:globals}\fi the volume
names, privileges and users are stored on all nodes. However the volume's
contents are stored only on a subset
of nodes:
\begin{description}
	\item[volA\_r1] is a volume with replica 1: its data and filenames
	are only stored in one place (no copies)
	\item[volB\_r2] is a volume with replica 2: the data is always stored
	on (at least) 2 distinct nodes, and its filenames are stored on exactly
	2 specific distinct nodes
	\item[volC\_r3] is a volume with replica 3: the data is always stored
	on (at least) 3 distinct nodes, and its filenames are stored on exactly
	3 specific distinct nodes
	\item and so on\ldots
\end{description}
The cluster \emph{can} have multiple volumes with different replica counts at
the same time. A volume can also have an arbitrary metadata attached to.


\section{Jobs}
Certain operations, such as finalizing a file upload (which may require
replication of data), can take a long time. To avoid blocking other operations,
all taks which involve more than one node create a job and the \SX clients poll
for the job's outcome instead of blocking and waiting for it to finish. This
allows to speed up recursive uploads for example: each file creates a new job
and the client only waits for completion at the end of the recursive upload.
The cluster tries its best to retry when transient errors occur internally.
In case of a failure, it will abort or undo the operation and report the status
to the client. The jobs are also used for conflict resolution: on conflicting
operations, for example creating two users with same name, only one job is
guaranteed to ``win'' and all the others will be aborted.

\chapter{Installation / Upgrade}

Skylable \SX is tested on all popular UNIX platforms, including Linux,
FreeBSD, and Mac OS X. We try to support as many platforms as possible,
if you have troubles installing, compiling or running our software on
your platform please let us know.

\section{Minimum requirements}
The default setup described in this manual requires 2GB of RAM available
for each node. \SX can also be installed on machines with lower resources,
such as embedded ARM devices, but that requires advanced configuration not
covered by the manual.

\section{Binary packages}
The binary packages are available for all popular Linux distributions
and this is the easiest and recommended way to install Skylable \SX.

\subsection{Debian Wheezy}
Add the following entry to \path{/etc/apt/sources.list}:
\begin{lstlisting}
deb http://cdn.skylable.com/debian wheezy main
\end{lstlisting}
then run the following commands:
\begin{lstlisting}
$ curl 'https://pgp.mit.edu/pks/lookup?op=get&search=0x5377E192B7BC1D2E' | sudo apt-key  add -
$ sudo apt-get update
$ sudo apt-get install sx
\end{lstlisting}

\subsection{CentOS 5/6/7}
Create the file \path{/etc/yum.repos.d/skylable-sx.repo} with this content:
\begin{lstlisting}
[skylable-sx]
name=Skylable SX
baseurl=http://cdn.skylable.com/centos/$releasever/$basearch
enabled=1
gpgcheck=0
\end{lstlisting}
then execute:
\begin{lstlisting}
# yum install skylable-sx
\end{lstlisting}

\subsection{Fedora 20}
Create the file \path{/etc/yum.repos.d/skylable-sx.repo} with this content:
\begin{lstlisting}
[skylable-sx]
name=Skylable SX
baseurl=http://cdn.skylable.com/fedora/$releasever/$basearch
enabled=1
gpgcheck=0
\end{lstlisting}
then execute:
\begin{lstlisting}
# yum install skylable-sx
\end{lstlisting}

\section{Source code}
In order to compile \SX from source, you will need the following packages to
be installed together with their development versions:
\begin{itemize}
    \item OpenSSL/NSS
    \item libcurl $\ge$ 7.34.0 (otherwise the embedded one will be used)
    \item zlib
\end{itemize}
For example, on Debian run:
\begin{lstlisting}
# apt-get install libssl-dev libcurl4-openssl-dev libz-dev
\end{lstlisting}

\subsection{Compilation}
The software is based on autoconf, and you can just perform the standard
installation steps. The following commands install all the software in
\path{/usr/local}:
\begin{lstlisting}
$ ./configure && make
# make install
\end{lstlisting}
The rest of the manual assumes that \SX was installed from a binary
package, so some paths may be different.

\section{Upgrade existing cluster}
To take advantage of new features and improvements, it's recommended to
keep the cluster software up to date. The upgrade procedure has been
simplified and automated as much as possible to allow a smooth update
of a live cluster.
\subsection{Upgrading a single node}
It is recommended to upgrade one node at a time. In case of a problem, the
other nodes will stay unaffected and will be able to serve data to the
clients.
First install the latest version of Skylable \SX in the same way as the
previous deployment. Then run the following command:
\begin{lstlisting}
# sxsetup --upgrade
Upgrading local node...
[sx_storage_upgrade]: Performing integrity check on /var/lib/sxserver/storage
[sx_storage_upgrade]: Integrity check completed in 0s
[upgrade_db]: Upgraded DB /var/lib/sxserver/storage from SX-Storage 1.6 to SX-Storage 1.7
[...]
[sx_storage_upgrade]: Successfully upgraded all DBs
[sx_storage_upgrade]: Committing changes
[sx_storage_upgrade]: Schema upgrade completed in 0s
[sx_storage_upgrade]: Storage closed in 1s
[upgrade_node]: Storage is up to date
Starting SX.fcgi
Starting sxhttpd
SX node started successfully

Could not upgrade the remote cluster data at this point. Please run
sxsetup --upgrade again when all other nodes are locally upgraded and
online. The cluster should also get upgraded automatically after the
last node is locally upgraded.
\end{lstlisting}
The above is the expected output when not all of the nodes have been
updated yet. When running \verb+sxsetup --upgrade+ on the last node
of the cluster, it should also start the upgrade of the remote cluster
data:
\begin{lstlisting}
# sxsetup --upgrade
Upgrading local node...
[sx_storage_upgrade]: Performing integrity check on /var/lib/sxserver/storage
[sx_storage_upgrade]: Integrity check completed in 0s
[upgrade_db]: Upgraded DB /var/lib/sxserver/storage from SX-Storage 1.6 to SX-Storage 1.7
[...]
[sx_storage_upgrade]: Successfully upgraded all DBs
[sx_storage_upgrade]: Committing changes
[sx_storage_upgrade]: Schema upgrade completed in 0s
[sx_storage_upgrade]: Storage closed in 1s
[upgrade_node]: Storage is up to date
Starting SX.fcgi
Starting sxhttpd
SX node started successfully

Upgrading remote cluster data...
Versions:
    192.168.1.101: SX-Storage 1.7 (1.1)
    192.168.1.102: SX-Storage 1.7 (1.1)
    192.168.1.103: SX-Storage 1.7 (1.1)
Triggering upgrade and garbage collector
Current configuration: 536870912000/192.168.100.1/d3f8ad83-d003-4aaa-bbfb-73359af85991 536870912000/192.168.100.2/abc2ed51-b4a8-46b6-a8ac-0beb58e697d2 536870912000/192.168.100.3/a343b7f9-0bef-4f03-8c6f-526ca12d75a9
Operating mode: read-write
State of operations:
  * node d3f8ad83-d003-4aaa-bbfb-73359af85991 (192.168.100.1): Upgrade - local file blocks: 2870963 remaining
  * node abc2ed51-b4a8-46b6-a8ac-0beb58e697d2 (192.168.100.2): Upgrade - local file blocks: 4111032 remaining
  * node a343b7f9-0bef-4f03-8c6f-526ca12d75a9 (192.168.100.3): Upgrade - local file blocks: 10438108 remaining
Distribution: 872eeecb-ebf9-4368-8150-beb23cd44edf(v.7) - checksum: 18024964248989723179
Cluster UUID: cc8ab859-619e-4806-ade6-c32ab2db1665

Remote upgrade started. You can monitor the progress by running:
/usr/sbin/sxadm cluster --info sx://admin@mycluster
\end{lstlisting}
The remote upgrade process can take some time, depending on the amount of
data in the cluster. The progress can be monitored with
\verb+sxadm cluster --info+ as instructed above. When it's done, running
\verb+sxsetup --upgrade+ on any node should report the cluster is up to date:
\begin{lstlisting}
# sxsetup --upgrade
Local node is up-to-date.
Versions:
    192.168.1.101: SX-Storage 1.7 (1.1)
    192.168.1.102: SX-Storage 1.7 (1.1)
    192.168.1.103: SX-Storage 1.7 (1.1)
Cluster already fully upgraded

All components of the cluster are up-to-date!
\end{lstlisting}

\chapter{Installation}

Skylable \SX is tested on all popular UNIX platforms, including Linux,
FreeBSD, and Mac OS X. We try to support as many platforms as possible,
if you have troubles installing, compiling or running our software on
your platform please let us know.

\section{Binary packages}
The binary packages are available for all popular Linux distributions
and this is the easiest and recommended way to install Skylable \SX.

\subsection{Debian Wheezy}
Add the following entry to \path{/etc/apt/sources.list}:
\begin{lstlisting}
deb http://cdn.skylable.com/debian wheezy main
\end{lstlisting}
then run the following commands:
\begin{lstlisting}
# wget https://pgp.mit.edu/pks/lookup?op=get&search=0x5377E192B7BC1D2E | sudo apt-key add -
# apt-get install sx
\end{lstlisting}

\subsection{CentOS 5/6/7}
Create the file \path{/etc/yum.repos.d/skylable-sx.repo} with this content:
\begin{lstlisting}
[skylable-sx]
name=Skylable SX
baseurl=http://cdn.skylable.com/centos/$releasever/$basearch
enabled=1
gpgcheck=0
\end{lstlisting}
then execute:
\begin{lstlisting}
# yum install skylable-sx
\end{lstlisting}

\subsection{Fedora 20}
Create the file \path{/etc/yum.repos.d/skylable-sx.repo} with this content:
\begin{lstlisting}
[skylable-sx]
name=Skylable SX
baseurl=http://cdn.skylable.com/fedora/$releasever/$basearch
enabled=1
gpgcheck=0
\end{lstlisting}
then execute:
\begin{lstlisting}
# yum install skylable-sx
\end{lstlisting}

\section{Source code}
In order to compile \SX from source, you will need the following packages to
be installed together with their development versions:
\begin{itemize}
    \item OpenSSL/NSS
    \item libcurl $\ge$ 7.34.0 (otherwise the embedded one will be used)
    \item zlib
\end{itemize}
For example, on Debian run:
\begin{lstlisting}
# apt-get install libssl-dev libcurl4-openssl-dev libz-dev
\end{lstlisting}

\subsection{Compilation}
The software is based on autoconf, and you can just perform the standard
installation steps. The following commands install all the software in
\path{/usr/local}:
\begin{lstlisting}
$ ./configure && make
# make install
\end{lstlisting}
The rest of the manual assumes that \SX was installed from a binary
package, so some paths may be different.

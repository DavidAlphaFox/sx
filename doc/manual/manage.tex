\chapter{Cluster Management}

\section{Local node status}
You can check status of a specific node by running \path{sxserver status} on
that node:
\begin{lstlisting}
# /opt/sx/sbin/sxserver status
--- SX STATUS ---
sx.fcgi is running (PID 14394)
sxhttpd is running (PID 14407)

--- SX INFO ---
Cluster name: ^\marked{mycluster}^
Cluster port: 443
HashFS Version: SX-Storage 1.5
Cluster UUID: 01dca714-8cc9-4e26-960e-daf04892b1e2
Cluster authentication: CLUSTER/ALLNODE/ROOT/USERwBdjfz3tKcnTF2ouWIkTipreYuYjAAA
Admin key: ^\marked{0DPiKuNIrrVmD8IUCuw1hQxNqZfIkCY+oKwxi5zHSPn5y0SOi3IMawAA}^
Internal cluster protocol: SECURE
Used disk space: 17568768
Actual data size: 463872
List of nodes:
         * ec4d9d63-9fa3-4d45-838d-3e521f124ed3 ^\marked{192.168.1.101}^ (192.168.1.101) 536870912000
Storage location: /opt/sx/var/lib/sxserver/data
SSL private key: /opt/sx/etc/ssl/private/sxkey.pem
SX Logfile: /opt/sx/var/log/sxserver/sxfcgi.log
\end{lstlisting}
This gives you the information about local services and disk usage, but
also provides the admin key, which is needed for accessing the cluster
itself.

\section{Administrator access}
During cluster deployment a default admin account gets created
and initialized. You should be able to access the cluster from
any node using \path{sx://admin@mycluster} profile. In order
to manage the cluster remotely or from another system account,
you need to initialize access to the cluster using \path{sxinit}. 
In the example below we use the default admin account created
during cluster setup. Since "mycluster" is not a DNS name, we need
to point sxinit to one of the nodes of the cluster. It will
automatically discover the IP addresses of the other nodes.
Additionally, we create an alias \path{@cladm}, which later
can be used instead of \path{sx://admin@mycluster}.
\begin{lstlisting}
$ sxinit -l 192.168.1.101 -A @cladm sx://admin@mycluster
Warning: self-signed certificate:

        Subject: C=GB, ST=UK, O=SX, CN=mycluster
	Issuer: C=GB, ST=UK, O=SX, CN=mycluster
	SHA1 Fingerprint: 84:EF:39:80:1E:28:9C:4A:C8:80:E6:56:57:A4:CD:64:2E:23:99:7A

Do you trust this SSL certificate? [y/N] ^\marked{y}^
Trusting self-signed certificate
Please enter the user key:
^\marked{0DPiKuNIrrVmD8IUCuw1hQxNqZfIkCY+oKwxi5zHSPn5y0SOi3IMawAA}^
\end{lstlisting}

\section{User management}
A new cluster has only a single admin user. It is recommended to only use the
admin account for administrative purposes and perform regular operations
as a normal user. Use \path{sxacl useradd} to add new users to the cluster:
\begin{lstlisting}
$ sxacl useradd joe @cladm
User successfully created!
Name: joe
Key : FqmlTd9CWZUuPBGMdjE46DaT1/3kx+EYbahlrhcdVpy/9ePfrtWCIgAA
Type: normal

Run 'sxinit sx://joe@mycluster' to start using the cluster as user 'joe'.
\end{lstlisting}
Existing users can be listed with:
\begin{lstlisting}
$ sxacl userlist @cladm
admin (admin)
joe (normal)
\end{lstlisting}
To retrieve the current authentication key for a user run:
\begin{lstlisting}
$ sxacl usergetkey joe @cladm
5tJdVr+RSpA/IPuFeSwUeePtKdbDLWUKqoaoZLkmCcXTw5qzPg5e7AAA
\end{lstlisting}
Finally, to permanently delete a user from the cluster run the following
command:
\begin{lstlisting}
$ sxacl userdel joe @cladm
User 'joe' successfully removed.
\end{lstlisting}
All volumes owned by the user will be reassigned to the cluster
administrator performing the removal.
